\documentclass[11pt,a4paper]{article}

\usepackage{../ve230}

\author{\href{liuyh615@sjtu.edu.cn}{Yihao Liu} (515370910207)}
\semester{Summer}
\year{2019}
\subtitle{Homework}
\subtitlenumber{3}
\blockinfo{}

\usetikzlibrary{positioning}

\begin{document}

\maketitle

\subsection{3-22}
\begin{enumerate}[label=\alph*)]
\item 
$$\rho_{ps}=\mathbf{P}\cdot\mathbf{a_n}|_{n=L/2}=\frac{1}{2}P_0L.$$
$$\rho_p=-\nabla\cdot\mathbf{P}=-3P_0.$$
\item
$$Q_s=\oint_S\rho_{ps}dS=\frac{1}{2}P_0L\cdot 6L^2=3P_0L^3,$$
$$Q_v=\int_v\rho_pdV=\int_{-L/2}^{L/2}\int_{-L/2}^{L/2}\int_{-L/2}^{L/2}\rho_pdzdydx=-3P_0L^3,$$
$$Q=Q_s+Q_v=0.$$
\end{enumerate}

\subsection{3-23}
Let $\mathbf{P}=\mathbf{a_p}P_0$, $\theta=\langle \mathbf{P},\mathbf{a_n} \rangle$,
$$\rho_{ps}(\theta)=\mathbf{P}\cdot\mathbf{a_n}=P_0\cos\theta,$$
$$dE_\theta=dv\cdot\frac{\rho_{ps}}{4\pi\varepsilon_0R^2}\cdot\cos\theta=2\pi R^2\sin\theta d\theta\cdot\frac{P_0\cos\theta}{4\pi\varepsilon_0R^2}\cdot\cos\theta=\frac{P_0\sin\theta\cos\theta^2}{2\varepsilon_0}d\theta.$$
$$|\mathbf{E}|=\int dE_\theta=\int_0^\pi\frac{P_0\sin\theta\cos\theta^2}{2\varepsilon_0}d\theta=\frac{P_0}{3\varepsilon_0},$$
$$\mathbf{E}=\mathbf{a_p}\frac{P_0}{3\varepsilon_0}=\frac{\mathbf{P}}{3\varepsilon_0}.$$

\subsection{3-25}
$$E_{2t}=E_{1t}=\mathbf{a_x}2y-\mathbf{a_y}3x.$$
Since $\rho_s=0$,
$$\varepsilon_{r1}E_{1n}=\varepsilon_{r2}E_{2n},$$
$$E_{2n}=\frac{\varepsilon_{r1}}{\varepsilon_{r2}}E_{1n}=\frac{2}{3}\cdot\mathbf{a_z}5=\mathbf{a_z}\frac{10}{3}.$$
$$E_2=E_{2t}+E_{2n}=\mathbf{a_x}2y-\mathbf{a_y}3x+\mathbf{a_z}\frac{10}{3}.$$
$$\mathbf{D_2}=\varepsilon_2\mathbf{E_2}=3\varepsilon_0\left(\mathbf{a_x}2y-\mathbf{a_y}3x+\mathbf{a_z}\frac{10}{3}\right).$$

\subsection{3-28}
Obviously, $\mathbf{E_3}$ is parallel to $\mathbf{E_2}$, so we only need to find $\varepsilon_{r2}$ so that $\mathbf{E_2}$ is parallel to  the x-axis.
$$E_{1t}=E_{2t}=-3.$$
$$\varepsilon_{r1}E_{1n}=\varepsilon_{r2}E_{2n},$$
$$E_{2n}=\frac{\varepsilon_{r1}}{\varepsilon_{r2}}E_{1n}=\frac{1}{\varepsilon_{r2}}\cdot5=\frac{5}{\varepsilon_{r2}}.$$
$$E_{2t}\cos\theta+E_{2n}\cos\theta=0,$$
$$-3+\frac{5}{\varepsilon_{r2}}=0,$$
$$\varepsilon_{r2}=\frac{5}{3}.$$

\subsection{3-32}
$$C_1=\frac{2\pi\varepsilon L}{\ln\frac{b}{r_i}}=\frac{2\pi\varepsilon_0\varepsilon_{r_1}L}{\ln\frac{b}{r_i}},$$
$$C_2=\frac{2\pi\varepsilon L}{\ln\frac{r_o}{b}}=\frac{2\pi\varepsilon_0\varepsilon_{r_2}L}{\ln\frac{r_o}{b}},$$
$$C=\frac{1}{\frac{1}{C_1}+\frac{1}{C_2}}=\frac{2\pi\varepsilon_0L}{\frac{1}{\varepsilon_{r_1}}\ln\frac{b}{r_i}+\frac{1}{\varepsilon_{r_3}}\ln\frac{r_o	}{b}},$$
$$\frac{C}{L}=\frac{2\pi\varepsilon_0}{\frac{1}{\varepsilon_{r_1}}\ln\frac{b}{r_i}+\frac{1}{\varepsilon_{r_3}}\ln\frac{r_o}{b}}.$$

\subsection{3-43}
$$dW_e=Vdq=\frac{q}{C}dq,$$
$$W_e=\int dW_e=\int_0^Q \frac{q}{C}dq=\frac{Q^2}{2C}.$$
Since $Q=CV$, we can also get
$$W_e=\frac{1}{2}CV^2=\frac{1}{2}QV.$$
\end{document}



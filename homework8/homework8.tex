\documentclass[11pt,a4paper]{article}

\usepackage{../ve230}

\author{\href{liuyh615@sjtu.edu.cn}{Yihao Liu} (515370910207)}
\semester{Summer}
\year{2019}
\subtitle{Homework}
\subtitlenumber{8}
\blockinfo{}

\usetikzlibrary{positioning}

\begin{document}

\maketitle

\subsection{7-24}
$$\nabla\times\mathbf{E}=-\frac{\partial\mathbf{B}}{\partial t}=-\mu\frac{\partial\mathbf{H}}{\partial t},$$
$$\nabla\times(\nabla\times\mathbf{E})=-\mu\frac{\partial}{\partial t}\left(\mathbf{J}+\frac{\partial\mathbf{D}}{\partial t}\right)=-\mu\frac{\partial\mathbf{J}}{\partial t}-\mu\varepsilon\frac{\partial^2\mathbf{E}}{\partial t^2},$$
$$\nabla\times(\nabla\times\mathbf{E})=\nabla(\nabla\cdot\mathbf{E})-\nabla^2\mathbf{E}=\frac{1}{\varepsilon}\nabla\rho-\nabla^2\mathbf{E},$$
$$\nabla^2\mathbf{E}=\frac{1}{\varepsilon}\nabla\rho+\mu\frac{\partial\mathbf{J}}{\partial t}+\mu\varepsilon\frac{\partial^2\mathbf{E}}{\partial t^2},$$
$$\nabla^2\mathbf{E}=\frac{1}{\varepsilon}\nabla\rho+\mu j\omega\mathbf{J}+\mu\varepsilon\omega^2\mathbf{E}.$$

$$\nabla\times\mathbf{H}=\mathbf{J}+\frac{\partial\mathbf{D}}{\partial t},$$
$$\nabla\times(\nabla\times\mathbf{H})=\nabla\times\mathbf{J}+\varepsilon\frac{\partial}{\partial t}(\nabla\times\mathbf{E})=\nabla\times\mathbf{J}-\mu\varepsilon\frac{\partial^2\mathbf{H}}{\partial t^2},$$
$$\nabla\times(\nabla\times\mathbf{H})=\nabla(\nabla\cdot\mathbf{H})-\nabla^2\mathbf{H}=-\nabla^2\mathbf{H},$$
$$\nabla^2\mathbf{H}=-\nabla\times\mathbf{J}+\mu\varepsilon\frac{\partial^2\mathbf{H}}{\partial t^2},$$
$$\nabla^2\mathbf{H}=-\nabla\times\mathbf{J}+\mu\varepsilon\omega^2\mathbf{H}.$$

\subsection{7-27}
\begin{align*}
\nabla\times\mathbf{E}&=\frac{1}{R^2\sin\theta}\begin{vmatrix}
\mathbf{a}_R & \mathbf{a}_\theta R & \mathbf{a}_\phi R\sin\theta \\
\frac{\partial}{\partial R} & \frac{\partial}{\partial\theta} & \frac{\partial}{\partial\phi} \\
0 & E_0\sin\theta e^{-jkR} & 0 \end{vmatrix}\\
&=\frac{1}{R^2\sin\theta}\left(\mathbf{a}_R\frac{\partial}{\partial\phi}E_0\sin\theta e^{-jkR}-\mathbf{a}_\phi R\sin\theta\frac{\partial}{\partial R}E_0\sin\theta e^{-jkR}\right)\\
&=\mathbf{a}_\phi\frac{-E_0jk\sin\theta e^{-jkR}}{R},
\end{align*}
$$\nabla\times\mathbf{E} = -\mu\frac{\partial\mathbf{H}}{\partial t}=-j\mu\omega\mathbf{H},$$
$$\mathbf{H}=\mathbf{a}_\phi\frac{E_0k\sin\theta e^{-jkR}}{\mu\omega R},\quad k=\omega\sqrt{\mu\varepsilon}.$$
$$\mathbf{H}(R)=\mathbf{a}_\phi\frac{E_0\sqrt{\mu\varepsilon}\sin\theta e^{-j\omega\sqrt{\mu\varepsilon}R}}{\mu R},$$
$$\mathbf{H}(R,t)=\text{Re}[\mathbf{H}(R)e^{j\omega t}]=\mathbf{a}_\phi\frac{E_0\sqrt{\mu\varepsilon}}{\mu R}\sin\theta\cos(\omega t-\omega\sqrt{\mu\varepsilon}R).$$

\subsection{7-29}

\begin{enumerate}[label=\alph*)]
\item
$$\nabla\times\mathbf{E}=-j\omega\mu_0\mathbf{H}=\omega^2\mu_0\varepsilon_0\nabla\times \pi_e,$$
$$\nabla\times(\mathbf{E}-\omega^2\mu_0\varepsilon_0\pi_e)=\mathbf{0},$$
$$\mathbf{E}=\omega^2\mu_0\varepsilon_0\pi_e+\mathbf{C}.$$
$$\nabla\times\mathbf{H}=j\omega\mathbf{D}=\omega^2\mu_0\varepsilon_0\nabla\times \pi_e,$$
$$\nabla\times(j\omega\varepsilon_0\nabla\times\pi_e)=j\omega\varepsilon_0(\omega^2\mu_0\varepsilon_0\pi_e+\frac{\mathbf{P}}{\varepsilon_0}+\mathbf{C}),$$
$$\nabla\times(\nabla\times\pi_e)=\nabla(\nabla\cdot\pi_e)-\nabla^2\pi_e=\omega^2\mu_0\varepsilon_0\pi_e+\frac{\mathbf{P}}{\varepsilon_0}+\mathbf{C},$$
$$\nabla^2\pi_e+\omega^2\mu_0\varepsilon_0\pi_e=\nabla(\nabla\cdot\pi_e)-\frac{\mathbf{P}}{\varepsilon_0}-\mathbf{C},$$
$$\mathbf{C}=\nabla(\nabla\cdot\pi_e),$$
$$\mathbf{E}=\omega^2\mu_0\varepsilon_0\pi_e+\nabla(\nabla\cdot\pi_e).$$
\item
$$k_0^2=\omega^2\mu_0\varepsilon_0,$$
$$\nabla^2\pi_e+k_0^2\pi_e=-\frac{\mathbf{P}}{\varepsilon_0}.$$
\end{enumerate}

\subsection{8-7}
Let $\phi=\omega_t-kz$,
$$\mathbf{E}(\phi)=\mathbf{a}_xE_{10}\sin\phi+\mathbf{a_y}E_{20}\sin(\phi+\psi),$$
$$\frac{E_x}{E_{10}}=\sin\phi,$$
$$\frac{E_y}{E_{20}}=\sin(\phi+\psi)=\sin\phi\cos\psi+\cos\phi\sin\psi=\frac{E_x}{E_{10}}\cos\psi+\sqrt{1-\left(\frac{E_x}{E_{10}}\right)^2}\sin\psi,$$
$$\left[\sqrt{1-\left(\frac{E_x}{E_{10}}\right)^2}\sin\psi\right]^2=\left(\frac{E_y}{E_{20}}\right)^2+\left(\frac{E_x}{E_{10}}\cos\psi\right)^2-2\left(\frac{E_y}{E_{20}}\right)\left(\frac{E_x}{E_{10}}\cos\psi\right),$$
$$\left(\frac{E_x}{E_{10}}\right)^2+\left(\frac{E_y}{E_{20}}\right)^2-2\frac{E_x}{E_{10}}\frac{E_y}{E_{20}}\cos\psi=\sin^2\psi,$$
$$\left(\frac{E_x}{E_{10}\sin\psi}\right)^2+\left(\frac{E_y}{E_{20}\sin\psi}\right)^2-2\frac{E_x}{E_{10}}\frac{E_y}{E_{20}}\frac{\cos\psi}{sin^2\psi}=1.$$

\subsection{8-9}
$$\nabla^2\mathbf{E}+k_c^2\mathbf{E}=0,$$
$$k_c=\omega\sqrt{\mu\varepsilon}=\beta-j\alpha,$$
$$\omega^2\mu\varepsilon=\omega^2\mu(\varepsilon-j\sigma/\omega)=\beta^2-\alpha^2-2j\alpha\beta,$$
$$\left\{\begin{aligned}
\beta^2+\alpha^2&=\omega^2\mu\sqrt{\varepsilon^2+\sigma^2/\omega^2}\\
\beta^2-\alpha^2&=\omega^2\mu\varepsilon\\
\end{aligned}\right.\Longrightarrow\left\{\begin{aligned}
\alpha&=\omega\sqrt{\frac{\mu\varepsilon}{2}}\left[\sqrt{1+\left(\frac{\sigma^2}{\omega\varepsilon}\right)^2}-1\right]^{1/2}\\
\beta&=\omega\sqrt{\frac{\mu\varepsilon}{2}}\left[\sqrt{1+\left(\frac{\sigma^2}{\omega\varepsilon}\right)^2}+1\right]^{1/2}
\end{aligned}\right..$$

\end{document}


